\documentclass[12pt,UTF8]{ctexbook}
\usepackage{ctex}
\usepackage{array}
\usepackage{graphicx}
\usepackage{wrapfig}
\usepackage[table,dvipsnames]{xcolor}
\usepackage{tabularx}
\usepackage{amsmath}
\usepackage{amssymb}
\usepackage{xfrac}
\usepackage{eucal}
\usepackage{titlesec}
\usepackage{amsthm}
\usepackage{tikz-cd}
\usepackage{enumitem}
\usepackage{verbatim}
\usepackage{fontspec,xunicode,xltxtra}
\usepackage{xeCJK} 

\definecolor{gl}{RGB}{246, 252, 240}
\definecolor{gd}{RGB}{236, 244, 230}
\definecolor{bg}{RGB}{242, 244, 228}


\setCJKmainfont[BoldFont=STZhongsong]{STSong}
\setCJKmonofont{simkai.ttf} % for \texttt
\setCJKsansfont{simfang.ttf} % for \textsf
\setlength\parskip{8pt}
\setlength{\fboxsep}{12pt}
\renewcommand\thesection{\arabic{chapter}.\arabic{section}}
\newtheorem{df}{定义}[section] 
\newtheorem{pp}{命题}[section]
\newtheorem{tm}{定理}[section]
\newtheorem{ex}{例子}[section]
\newtheorem{et}{例题}[section]
\newtheorem{sk}{思考}[section]
\newtheorem{po}{公理}
\newtheorem*{so}{解答}
\renewcommand\parallel{\mathrel{/\mskip-4mu/}}
\newcommand\dangle{%
    \mathord{\text{%
        \tikz[baseline] \draw (0.8em,0ex) -- (0.3em, 0ex) -- (.6em, 1.5ex) -- (.8em, 1.5ex) -- (.5em, 0ex) -- cycle;}}}
\newcommand\xangle{%
    \mathord{\text{%
        \tikz[baseline] \draw (0.8em,1.5ex) -- (0.3em, 0ex) -- (.64em, 0ex) -- (.8em, .36ex) -- (.42em, .36ex) -- cycle;}}}\newenvironment{proof2}{\paragraph{\textbf{证明:}}}{\hfill$\square$}
\newtheorem{xt}{习题}[section]
\newtheorem{cor}{推论}[pp]
% 列举环境的行间距
\setenumerate[1]{itemsep=0pt,partopsep=0pt,parsep=0pt,topsep=0pt}
\setitemize[1]{itemsep=0pt,partopsep=0pt,parsep=0pt,topsep=0pt}
\setdescription{itemsep=0pt,partopsep=0pt,parsep=0pt,topsep=0pt}
% 章节字体大小
\titleformat{\section}{\zihao{-2}\bfseries}{ \thesection }{16pt}{}
% 封面
\title{\zihao{0} \bfseries 第二册}
\author{\zihao{2} \texttt{大青花鱼}}
% \date{\bfseries\today}
\date{}
% 正文
\begin{document}
\maketitle
\tableofcontents
\newpage

\chapter{空间中的形状}

我们已经通过公理体系研究过平面中的简单形状,将基本的平面形和函数的图像联系起来,并且引入了向量的概念。
现在,我们进一步研究立体空间中的形状。人类生活在立体空间中,因此,研究空间中形体的性质,对我们认识世界、
改造世界有直接帮助。

\section{点、直线、平面}

和平面形一样,立体空间中的形也是从种种事物的形状总结提炼而来。平面是人类最早总结出的概念之一。
我们已经研究过平面中的形状,因此,研究立体空间时,我们把平面作为地位和点、直线相同的基本概念。

研究平面形状时,我们首先引进了公理体系。如今我们将平面公理体系扩展为立体空间的公理体系。
为此,我们要通过公设和公理定义\textbf{平面}以及它和点、直线的关系。

我们定义面为点的集合,也是线运动的结果。平面是最基本的面,一般用小写希腊字母$\alpha,\beta,\gamma$等表示。
\begin{po}{\textbf{平面公理}}\label{po:0}
    过不共线的三点,有且仅有一个平面。
\end{po}

我们也说三角形(或圆)确定一个平面。不共线的三点$A$、$B$、$C$确定的平面,可以记作平面$ABC$。

如何确定平面是“平”的呢?生活和生产中,我们一般用直线来确定一个面是不是平的。
比如,木工常常用角尺的直角边放在刨好的木板上。如果直角边总能与板面紧密贴合,就说明木板已刨平了。
水泥工用直的刮子将刚铺水泥的地面刮平。我们把人们总结出的经验作为判断平面的方法,用公理的形式确定下来。
\begin{po}{\textbf{平直公理}}\label{po:1}
    过平面上不重合两点的直线,在平面中。
\end{po}
平直公理说明,直线要么与平面没有公共点,要么只交于一点,要么全部在平面里。
直线与平面没有公共点,则称直线与平面平行,也用$\parallel$表记;直线与平面恰有一公共点,
则称直线与平面相交;直线与平面有两个或以上公共点,则称直线在平面中或平面经过直线。
用集合的语言来说,直线$l$在平面$\gamma$中,就说明$l$是$\gamma$的子集。

% 我们知道,直线把平面分为两部分,称为直线的两侧。空间中,我们把每侧称为半平面。
% 直线及其一侧的半平面构成一个展面。而每个平面又把空间分为两部分,称为平面的两方。根据实际需要,我们可以称一点在平面的上方或下方、前方或后方、左方或右方,等等。

从平面公理和平直公理,容易得到另两种定义平面的方法:
\begin{tm}\label{tm:1-0-0}
    过一条直线与该直线外一点,有且仅有一个平面。
\end{tm}
\begin{proof2}
    设直线为$l$,$P$为$l$外一点。取$l$上不同的两点,和$P$构成不共线的三点。这三点确定一个平面。
    根据平直公理,$l$在该平面中。
\end{proof2}
\begin{tm}\label{tm:1-0-1}
    过两条相交的直线,有且仅有一个平面。
\end{tm}
\begin{proof2}
    设有直线$l,m$。如果$l,m$相交,设交点为$P$,在$l,m$上各取不同于$P$的一点:$Q,R$,则$P,Q,R$不共线,
    于是确定一个平面$PQR$,根据平直公理,$l,m$都在$PQR$中。
\end{proof2}

设直线$l$与平面$\gamma$平行,没有公共点,那么它与$\gamma$中任何直线没有公共点。
比如,给定$\gamma$中一点$A$,$\gamma$中经过$A$的任何直线,都与$l$没有公共点。

平面中,两直线没有公共点,就说它们相互平行。空间中,两直线除了重合、相交和平行,还有另一种关系,
我们称之为直线\textbf{异面}。前面的例子中,根据平行公理,过$\gamma$中的点$A$,
恰有一条直线与$l$平行,其余与$l$不相交的直线,都称与$l$异面。

哪条直线与$l$平行呢?显然,在空间中,我们需要补充平行公理:
\begin{po}{\textbf{平行公理}}\label{po:2}
    过直线外一点,有且仅有一条与它平行的直线。它在直线与点确定的平面上。
\end{po}
新版的平行公理在原来的基础上,指定了平行线的位置:在直线与点确定的平面上。
换句话说,平行是一个平面内性质。两直线平行的关系必然发生在同一平面中。
也正因如此,我们把其它的无公共点的情形叫做异面。

从补充的平行公理出发,可以得到另一种定义平面的方法:
\begin{tm}\label{tm:1-0-2}
    过两条平行的直线,有且仅有一个平面。
\end{tm}
\begin{proof2}
    设直线$l \parallel m$,在$l$上找两点$P_1,P_2$,在$m$上找一点$Q$。$P_1P_2Q$确定唯一平面$\gamma$。
    在平面$\gamma$中,过$Q$作$l$的平行线。根据平行公理,这条平行线就是$m$。
    因此$l,m$共面,它们确定唯一平面$\gamma$。
\end{proof2}

再来看两个平面的关系。两个平面相交,交集是什么呢?生活中的经验告诉我们,两个平面相交,交集是直线。
比如,裁纸刀的刀面是平的,切在纸上,将纸面分成两部分,裁痕是直的。我们把这个性质用公理描述为:
\begin{po}{\textbf{交面公理}}\label{po:3}
    两个平面如果有交集,则交集至少包含两个不重合的点。
\end{po}
交面公理说明,两个平面不可能只交于一点。如果它们有两个(不重合的)公共点,那么根据平直公理,
两平面的交集包括过这两点的直线。
而如果两平面的交集中还有不属于这条直线的第三点,那么根据平面公理,这三点确定一个平面,于是两平面重合。

综上所述,从交面公理可以推出:要么两平面没有公共点,要么交集为一条直线$l$,称为两平面相交于直线$l$;
要么两平面重合。

有了交面公理,我们可以证明空间中平行直线的递推性质。由于证明繁琐,有些地方把它当作公理引入。
\begin{tm}\label{tm:1-0-10}
    如果直线$l_1,l_2$都平行于直线$m$,那么$l_1,l_2$平行或重合。
\end{tm}
\begin{proof2}
    设$l_1,m$确定的平面为$\gamma_1$,$l_2,m$确定的平面为$\gamma_2$。分两种情况讨论:
    

    1. $\gamma_1, \gamma_2$重合。那么$l_1,l_2,m$都在此平面上。根据平面直线的结论,$l_1,l_2$平行或重合。
    
    2. $\gamma_1, \gamma_2$不重合。于是$\gamma_1\cap \gamma_2 = m$。
    取$l_1$上一点$P$,$P$与$l_2$确定平面$\beta$。$l_2\notin\gamma_1$,所以$\beta,\gamma_1$不重合。
    设$\beta\cap\gamma_1 = n$。
    
    首先证明$n\parallel l_2$。反设$n$交$l_2$于点$Q$,则$Q\in\gamma_1\cap\gamma_2=m$,
    于是$Q\in m\cap l_2$。这与$m\parallel l_2$矛盾。因此$n\parallel l_2$。
    
    再证明$n\parallel m$。反设$n$交$m$于点$Q$,则过$Q$有$m\parallel l_2$和$n\parallel l_2$,于是$n=m$。
    但$P\in n$,因此$P\in m\cap l_1$,这与$m\parallel l_1$矛盾。因此$n\parallel m$。
    
    $n$与$l_1$都平行于$m$,且有公共点$P$,所以$n = l_1$。所以$l_1\parallel l_2$。
\end{proof2}

接下来回顾直线与平面的关系。从补充后的平行公理,可以得出直线与平面平行的判定方法:

\begin{tm}\label{tm:1-0-20}
    如果平面$\alpha$中有直线$m$平行于直线$l$,那么$l$平行于平面$\alpha$或在$\alpha$中。
\end{tm}
\begin{proof2}
    设直线$l$平行于平面$\alpha$中的某直线$m$。记$l,m$确定的平面为$\beta$。则要么$\alpha = \beta$,要么$\alpha\cap\beta = m$。如果$\alpha=\beta$,那么$l\subset\alpha$。如果$\alpha\cap\beta = m$,那么$\alpha\cap l= \beta\cap \alpha\cap l = m\cap l = \varnothing $,即$\alpha \parallel l$。
\end{proof2}
\begin{tm}\label{tm:1-0-30}
    如果直线$l$与平面$\alpha$平行,那么经过$l$的任意平面,若与$\alpha$相交,其交线也与$l$平行。
\end{tm}
\begin{proof2}
    设经过$l$的平面$\gamma$与$\alpha$交于直线$m$。一方面,$m,l$共面;另一方面$m\in\alpha$,因此$m,l$无公共点。这说明$m\parallel l$。
\end{proof2}
从这些结论可以继续推出:
\begin{tm}\label{tm:1-0-40}
    若直线$l$与平面$\alpha$平行,过$\alpha$中任一点作$l$的平行线$m$,则$m$在平面$\alpha$中。
\end{tm}
\begin{proof2}
    若直线$l$与平面$\alpha$平行,设它和$\alpha$中任一点$P$确定的平面为$\beta$,
    则$\beta$与$\alpha$相交。设交线为$m$,则$P\in m$。根据定理\ref{tm:1-0-30},$l\parallel m$。
    这就说明,过$P$而平行于$l$的直线在$\alpha$中。
\end{proof2}
\begin{tm}\label{tm:1-0-41}
    直线$l$与直线$m$平行,则过$m$的平面要么与$l$平行,要么经过$l$。
\end{tm}
\begin{proof2}
    给定过$m$的平面$\alpha$,$m\subset \alpha$与$l$平行,所以根据定理\ref{tm:1-0-20},
    $l$平行于平面$\alpha$或在$\alpha$中。
\end{proof2}

直线与平面的平行关系,也有传递性。
\begin{tm}
    若直线$l_1$与直线$l_2$平行,且与平面$\gamma$平行,则$l_2$与$\gamma$平行或在$\gamma$中。
\end{tm}
\begin{proof2}
    $l_1\parallel \gamma$。过$\gamma$中任一点作直线$n\parallel l_1$,则根据定理\ref{tm:1-0-40},
    $n$在$\gamma$中。$n\parallel l_1$,$l_1\parallel l_2$,所以根据定理\ref{tm:1-0-10},$n\parallel l_2$。
    根据定理\ref{tm:1-0-20},$l_2\parallel \gamma$或在$\gamma$中。
\end{proof2}

上面提到,两平面要么无公共点,要么相交,要么重合。我们把无公共点的平面称为平行平面,也用$\parallel$表记。
平面中,平行公理告诉我们,过直线外一点,恰有一条直线与之平行。空间中的平面,也有类似的结论:
\begin{tm}
    过平面外一点,恰有一平面与之平行。
\end{tm}\label{tm:1-0-50}
\begin{proof2}
    设平面$\alpha$外有点$P$。在$\alpha$上选一点$A$,过$A$作两相交直线$l,m$(交点为$A$)。
    根据平行公理,过$P$恰有直线$l', m'$分别与$l,m$平行。$l',m'$相交于点$P$,确定平面$\alpha'$。
    下面证明$\alpha,\alpha'$无公共点,即$\alpha'\parallel\alpha$。

    反设$\alpha,\alpha'$有公共点。由于$P\in\alpha'$在$\alpha$外,两者不重合。
    因此根据交面公理,$\alpha \cap \alpha'$是一条直线,记为$n$。$l,m,n$共面,$l,m$相交,
    因此$l,m$中至少有一条与$n$相交。设$l$与$n$相交,交点为$Q$,则$Q\in n\subset\alpha'$。又因为$l\parallel l'$,$Q\in l$,所以$Q\notin l'$,
    因而在$\alpha'$中,过$Q$可作$l'$的平行线。但这条线在$\alpha'$中,因此不是$l$。这与平行公理矛盾。
        
    因此,$\alpha,\alpha'$无公共点,$\alpha'\parallel\alpha$。
\end{proof2}

类似的结论还有:

\begin{tm}\label{tm:1-0-60}
    过平行于平面$\alpha$的一直线$l$,恰有一平面与$\alpha$平行。
\end{tm}
\begin{proof2}
在$l$上任取一点$P$,$P\notin \alpha$,因此根据定理\ref{tm:1-0-50},过$P$恰有一平面$\beta$与$\alpha$平行,只需证明$\beta$经过$l$。在$\alpha$中任取一点$Q$,则根据定理\ref{tm:1-0-40},过$Q$平行于$l$的直线$m$在$\alpha$中。$\beta$与$\alpha$平行,也就是说$\beta$与$\alpha$无公共点,所以$\beta$与$\alpha$的子集$m$也无公共点,即$m\parallel \beta$。过$P$作$m$的平行线,则根据定理\ref{tm:1-0-40},平行线在$\beta$中。而这条平行线就是$l$,所以$l$在$\beta$中。这说明过$l$恰有一平面$\beta$与$\alpha$平行。
\end{proof2}

从证明中,我们还可以提炼出判定平面平行(或重合)的准则:

\begin{tm}\label{tm:1-0-70}
    给定平面$\gamma_1, \gamma_2$。设$l,m$为$\gamma_1$中的相交直线。
    若$\gamma_2$中有直线$l',m'$分别与$l,m$平行或重合,则平面$\gamma_1, \gamma_2$平行或重合。
\end{tm}
\begin{proof2}
    两平面要么相互平行,要么重合,要么相交于一直线。反设$\gamma_1, \gamma_2$相交于直线$n$。

    如果$l=l'$,那么$l\subset \gamma_1\cap\gamma_2$,于是$n=l=l'$。设$m,l$交于点$P$,$m',l'$交于点$Q$。如果$P=Q$,那么$m=m'$,于是$\gamma_1$、$\gamma_2$都是$l,m$确定的平面,$\gamma_1=\gamma_2$。如果$P\neq Q$,那么$P\notin m'$。但$P\in l=n\subset\gamma_2$,因此过$P$作$m'$的平行线,平行线应该在$\gamma_2$中,因此根据平行公理,$m$在$\gamma_2$中。这说明$\gamma_1$、$\gamma_2$都是$l,m$确定的平面,$\gamma_1=\gamma_2$。于是总有两平面重合,矛盾。

    如果$l\parallel l'$,由于$l,m,n$共面,且$l,m$相交,
    因此$l,m$中至少有一条与$n$相交。设$l$与$n$相交,交点为$Q$,则$Q\in n\subset\gamma_2$。又因为$l\parallel l'$,$Q\in l$,所以$Q\notin l'$。
    在$\gamma_2$中,过$Q$可作$l'$的平行线。但这条线在$\gamma_2$中,因此不是$l$。这与平行公理矛盾。

    因此,平面$\gamma_1, \gamma_2$平行或重合。
\end{proof2}
平行平面之间,也有类似平行直线的传递性。
\begin{tm}\label{tm:1-0-80}

如果平面$\gamma_1,\gamma_2$都平行于平面$\beta$,那么$\gamma_1,\gamma_2$平行或重合。
\end{tm}
我们先证明一个小结论:
\begin{tm}\label{tm:1-0-90}
    设平面$\gamma_1\parallel \gamma_2$。平面$\beta$与$\gamma_1,\gamma_2$相交于直线$l_1,l_2$,
    则$l_1\parallel l_2$。
\end{tm}
\begin{proof2}
    一方面,$l_1,l_2$共面。另一方面,$\gamma_1\parallel \gamma_2$说明$l_1,l_2$无公共点。
    所以$l_1\parallel l_2$。
\end{proof2}
从这个结论还可以推出:如果平面$\gamma_1\parallel \gamma_2$,那么对$\gamma_1$中任意直线,过$\gamma_2$中任一点,作它的平行线,平行线都在$\gamma_2$中。

再来证明定理\ref{tm:1-0-80}。
\begin{proof2}
    已知平面$\gamma_1,\gamma_2$都平行于平面$\beta$。在$\beta$中找一点$P$,过$P$作相交直线$l,m$。在$\gamma_1$中找一点$Q$,过$Q$分别作$l,m$的平行线$l_1,m_1$,则$l_1,m_1$都在$\gamma_1$中。它们分别是平面$\gamma_1$与$l,Q$确定的平面$\alpha_1$、平面$\gamma_1$与$m,Q$确定的平面$\alpha_2$的交线。
    设$\alpha_1,\alpha_2$分别与$\gamma_2$交于$l_2,m_2$,由于$\gamma_2\parallel \beta$,
    所以根据定理\ref{tm:1-0-30},$l_2\parallel l$,$m_2\parallel m$。因此根据定理\ref{tm:1-0-10},
    $l_2$与$l_1$平行或重合,$m_2$与$m_1$平行或重合。根据定理\ref{tm:1-0-70},$\gamma_1$、$\gamma_2$平行或重合。
\end{proof2}

最后,我们还可以得到:
\begin{tm}\label{tm:1-0-100}
    若平面$\gamma_1$与直线$l$平行,且与平面$\gamma_2$平行,则$l$与$\gamma_2$平行或在$\gamma_2$中。
\end{tm}
\begin{proof2}
    $l\parallel \gamma_1$,所以过$l$恰有一平面$\beta$与$\gamma_1$平行。如果$\beta=\gamma_2$,
    则$l\subset\gamma_2$。如果$\beta\parallel\gamma_2$,那么$l\parallel \gamma_2$。
    如果$\beta$与$\gamma_2$相交于直线$m$,那么由于$m,l$共面且无公共点,$m\parallel l$。
    于是,根据定理\ref{tm:1-0-20},$l\parallel \gamma_2$或在$\gamma_2$中。
\end{proof2}

总结:

我们初步建立了关于空间形状的公理体系,引入了空间中平面的概念,并界定了点、直线和平面的关系:
\begin{enumerate}
    \item 直线和平面都是点的集合。
    \item 直线可能与平面平行、相交,或在平面中。
    \item 直线可能与直线异面、平行、相交、重合。
    \item 平面可能与平面平行、相交、重合。
    \item 直线与直线相交于一点,直线与平面相交于一点,平面与平面相交于一直线。
\end{enumerate}


\begin{sk}
    \mbox{} \\
    \indent 1. 定理\ref{tm:1-0-100}的证明中,我们讨论了$\beta$与$\gamma_2$相交于直线$m$的情形。实际上$\beta$是否会与$\gamma_2$相交?如何看待这个论证?
\end{sk}

\begin{xt}
    \mbox{} \\
    \indent 1. 如果直线$l$与直线$m$平行,那么过$l$的平面与过$m$的平面要么平行,要么重合,
    要么交于$l,m$之一,要么交于与$l,m$都平行的直线$n$。 
\end{xt}


\section{空间向量}

上一节中,我们使用公理体系讨论空间中的形状,







\end{document}