\documentclass[12pt,UTF8]{ctexbook}
\usepackage[toc,page]{appendix}
\usepackage{ctex}
\usepackage{array}
\usepackage{graphicx}
\usepackage{wrapfig}
\usepackage[table,dvipsnames]{xcolor}
\usepackage{tabularx}
\usepackage{amsmath}
\usepackage{amssymb}
\usepackage{xfrac}
\usepackage{eucal}
\usepackage{titlesec}
\usepackage{amsthm}
\usepackage{tikz-cd}
\usepackage{enumitem}
\usepackage{verbatim}
\usepackage{fontspec,xunicode,xltxtra}
\usepackage{xeCJK} 

\definecolor{gl}{RGB}{246, 252, 240}
\definecolor{gd}{RGB}{236, 244, 230}
\definecolor{bg}{RGB}{242, 244, 228}


\setCJKmainfont[BoldFont=STZhongsong]{STSong}
\setCJKmonofont{simkai.ttf} % for \texttt
\setCJKsansfont{simfang.ttf} % for \textsf
\setlength\parskip{8pt}
\setlength{\fboxsep}{12pt}
\renewcommand\thesection{\arabic{chapter}.\arabic{section}}
\newtheorem{df}{定义}[section] 
\newtheorem{pp}{命题}[section]
\newtheorem{tm}{定理}[section]
\newtheorem{ex}{例子}[section]
\newtheorem{et}{例题}[section]
\newtheorem{sk}{思考}[section]
\newtheorem{po}{公理}
\newtheorem*{so}{解答}
\renewcommand\parallel{\mathrel{/\mskip-4mu/}}
\newcommand\dangle{%
    \mathord{\text{%
        \tikz[baseline] \draw (0.8em,0ex) -- (0.3em, 0ex) -- (.6em, 1.5ex) -- (.8em, 1.5ex) -- (.5em, 0ex) -- cycle;}}}
\newcommand\xangle{%
    \mathord{\text{%
        \tikz[baseline] \draw (0.8em,1.5ex) -- (0.3em, 0ex) -- (.64em, 0ex) -- (.8em, .36ex) -- (.42em, .36ex) -- cycle;}}}\newenvironment{proof2}{\paragraph{\textbf{证明:}}}{\hfill$\square$}
\newtheorem{xt}{习题}[section]
\newtheorem{cor}{推论}[pp]
% 列举环境的行间距
\setenumerate[1]{itemsep=0pt,partopsep=0pt,parsep=0pt,topsep=0pt}
\setitemize[1]{itemsep=0pt,partopsep=0pt,parsep=0pt,topsep=0pt}
\setdescription{itemsep=0pt,partopsep=0pt,parsep=0pt,topsep=0pt}
% 章节字体大小
\titleformat{\section}{\zihao{-2}\bfseries}{ \thesection }{16pt}{}
% 封面
\title{\zihao{0} \bfseries 第二册}
\author{\zihao{2} \texttt{大青花鱼}}
% \date{\bfseries\today}
\date{}
% 正文
\begin{document}
\maketitle
\tableofcontents
\newpage

\chapter{空间中的形状}

我们已经通过公理体系研究过平面中的简单形状,将基本的平面形和函数的图像联系起来,并且引入了向量的概念。
现在,我们进一步研究立体空间中的形状。人类生活在立体空间中,因此,研究空间中形体的性质,对我们认识世界、
改造世界有直接帮助。

\section{点、直线、平面}

和平面形一样,立体空间中的形也是从种种事物的形状总结提炼而来。平面是人类最早总结出的概念之一。
我们已经研究过平面中的形状,因此,研究立体空间时,我们把平面作为地位和点、直线相同的基本概念。

研究平面形状时,我们首先引进了公理体系。如今我们将平面公理体系扩展为立体空间的公理体系。
为此,我们要通过公设和公理定义\textbf{平面}以及它和点、直线的关系。

我们定义面为点的集合,也是线运动的结果。平面是最基本的面,一般用小写希腊字母$\alpha,\beta,\gamma$等表示。
\begin{po}{\textbf{平面公理}}\label{po:0}
    过不共线的三点,有且仅有一个平面。
\end{po}

我们也说三角形(或圆)确定一个平面。不共线的三点$A$、$B$、$C$确定的平面,可以记作平面$ABC$。

如何确定平面是“平”的呢?生活和生产中,我们一般用直线来确定一个面是不是平的。
比如,木工常常用角尺的直角边放在刨好的木板上。如果直角边总能与板面紧密贴合,就说明木板已刨平了。
水泥工用直的刮子将刚铺水泥的地面刮平。我们把人们总结出的经验作为判断平面的方法,用公理的形式确定下来。
\begin{po}{\textbf{平直公理}}\label{po:1}
    过平面上不重合两点的直线,在平面中。
\end{po}
平直公理说明,直线要么与平面没有公共点,要么只交于一点,要么全部在平面里。
直线与平面没有公共点,则称直线与平面平行,也用$\parallel$表记;直线与平面恰有一公共点,
则称直线与平面相交;直线与平面有两个或以上公共点,则称直线在平面中或平面经过直线。
用集合的语言来说,直线$l$在平面$\gamma$中,就说明$l$是$\gamma$的子集。

% 我们知道,直线把平面分为两部分,称为直线的两侧。空间中,我们把每侧称为半平面。
% 直线及其一侧的半平面构成一个展面。而每个平面又把空间分为两部分,称为平面的两方。根据实际需要,我们可以称一点在平面的上方或下方、前方或后方、左方或右方,等等。

从平面公理和平直公理,容易得到另两种定义平面的方法:
\begin{tm}\label{tm:1-0-0}
    过一条直线与该直线外一点,有且仅有一个平面。
\end{tm}
\begin{proof2}
    设直线为$l$,$P$为$l$外一点。取$l$上不同的两点,和$P$构成不共线的三点。这三点确定一个平面。
    根据平直公理,$l$在该平面中。
\end{proof2}
\begin{tm}\label{tm:1-0-1}
    过两条相交的直线,有且仅有一个平面。
\end{tm}
\begin{proof2}
    设有直线$l,m$。如果$l,m$相交,设交点为$P$,在$l,m$上各取不同于$P$的一点:$Q,R$,则$P,Q,R$不共线,
    于是确定一个平面$PQR$,根据平直公理,$l,m$都在$PQR$中。
\end{proof2}

设直线$l$与平面$\gamma$平行,没有公共点,那么它与$\gamma$中任何直线没有公共点。
比如,给定$\gamma$中一点$A$,$\gamma$中经过$A$的任何直线,都与$l$没有公共点。

平面中,两直线没有公共点,就说它们相互平行。空间中,两直线除了重合、相交和平行,还有另一种关系,
我们称之为直线\textbf{异面}。前面的例子中,根据平行公理,过$\gamma$中的点$A$,
恰有一条直线与$l$平行,其余与$l$不相交的直线,都称与$l$异面。

哪条直线与$l$平行呢?显然,在空间中,我们需要补充平行公理:
\begin{po}{\textbf{平行公理}}\label{po:2}
    过直线外一点,有且仅有一条与它平行的直线。它在直线与点确定的平面上。
\end{po}
新版的平行公理在原来的基础上,指定了平行线的位置:在直线与点确定的平面上。
换句话说,平行是一个平面内性质。两直线平行的关系必然发生在同一平面中。
也正因如此,我们把其它的无公共点的情形叫做异面。

从补充的平行公理出发,可以得到另一种定义平面的方法:
\begin{tm}\label{tm:1-0-2}
    过两条平行的直线,有且仅有一个平面。
\end{tm}
\begin{proof2}
    设直线$l \parallel m$,在$l$上找两点$P_1,P_2$,在$m$上找一点$Q$。$P_1P_2Q$确定唯一平面$\gamma$。
    在平面$\gamma$中,过$Q$作$l$的平行线。根据平行公理,这条平行线就是$m$。
    因此$l,m$共面,它们确定唯一平面$\gamma$。
\end{proof2}

再来看两个平面的关系。两个平面相交,交集是什么呢?生活中的经验告诉我们,两个平面相交,交集是直线。
比如,裁纸刀的刀面是平的,切在纸上,将纸面分成两部分,裁痕是直的。我们把这个性质用公理描述为:
\begin{po}{\textbf{交面公理}}\label{po:3}
    两个平面如果有交集,则交集至少包含两个不重合的点。
\end{po}
交面公理说明,两个平面不可能只交于一点。如果它们有两个(不重合的)公共点,那么根据平直公理,
两平面的交集包括过这两点的直线。
而如果两平面的交集中还有不属于这条直线的第三点,那么根据平面公理,这三点确定一个平面,于是两平面重合。

综上所述,从交面公理可以推出:要么两平面没有公共点,要么交集为一条直线$l$,称为两平面相交于直线$l$;
要么两平面重合。

有了交面公理,我们可以证明空间中平行直线的递推性质。由于证明繁琐,有些地方把它当作公理引入。
\begin{tm}\label{tm:1-0-10}
    如果直线$l_1,l_2$都平行于直线$m$,那么$l_1,l_2$平行或重合。
\end{tm}
\begin{proof2}
    设$l_1,m$确定的平面为$\gamma_1$,$l_2,m$确定的平面为$\gamma_2$。分两种情况讨论:
    

    1. $\gamma_1, \gamma_2$重合。那么$l_1,l_2,m$都在此平面上。根据平面直线的结论,$l_1,l_2$平行或重合。
    
    2. $\gamma_1, \gamma_2$不重合。于是$\gamma_1\cap \gamma_2 = m$。
    取$l_1$上一点$P$,$P$与$l_2$确定平面$\beta$。$l_2\notin\gamma_1$,所以$\beta,\gamma_1$不重合。
    设$\beta\cap\gamma_1 = n$。
    
    首先证明$n\parallel l_2$。反设$n$交$l_2$于点$Q$,则$Q\in\gamma_1\cap\gamma_2=m$,
    于是$Q\in m\cap l_2$。这与$m\parallel l_2$矛盾。因此$n\parallel l_2$。
    
    再证明$n\parallel m$。反设$n$交$m$于点$Q$,则过$Q$有$m\parallel l_2$和$n\parallel l_2$,于是$n=m$。
    但$P\in n$,因此$P\in m\cap l_1$,这与$m\parallel l_1$矛盾。因此$n\parallel m$。
    
    $n$与$l_1$都平行于$m$,且有公共点$P$,所以$n = l_1$。所以$l_1\parallel l_2$。
\end{proof2}

接下来回顾直线与平面的关系。从补充后的平行公理,可以得出直线与平面平行的判定方法:

\begin{tm}\label{tm:1-0-20}
    如果平面$\alpha$中有直线$m$平行于直线$l$,那么$l$平行于平面$\alpha$或在$\alpha$中。
\end{tm}
\begin{proof2}
    设直线$l$平行于平面$\alpha$中的某直线$m$。记$l,m$确定的平面为$\beta$。则要么$\alpha = \beta$,要么$\alpha\cap\beta = m$。如果$\alpha=\beta$,那么$l\subset\alpha$。如果$\alpha\cap\beta = m$,那么$\alpha\cap l= \beta\cap \alpha\cap l = m\cap l = \varnothing $,即$\alpha \parallel l$。
\end{proof2}
\begin{tm}\label{tm:1-0-30}
    如果直线$l$与平面$\alpha$平行,那么经过$l$的任意平面,若与$\alpha$相交,其交线也与$l$平行。
\end{tm}
\begin{proof2}
    设经过$l$的平面$\gamma$与$\alpha$交于直线$m$。一方面,$m,l$共面;另一方面$m\in\alpha$,因此$m,l$无公共点。这说明$m\parallel l$。
\end{proof2}
从这些结论可以继续推出:
\begin{tm}\label{tm:1-0-40}
    若直线$l$与平面$\alpha$平行,过$\alpha$中任一点作$l$的平行线$m$,则$m$在平面$\alpha$中。
\end{tm}
\begin{proof2}
    若直线$l$与平面$\alpha$平行,设它和$\alpha$中任一点$P$确定的平面为$\beta$,
    则$\beta$与$\alpha$相交。设交线为$m$,则$P\in m$。根据定理\ref{tm:1-0-30},$l\parallel m$。
    这就说明,过$P$而平行于$l$的直线在$\alpha$中。
\end{proof2}
\begin{tm}\label{tm:1-0-41}
    直线$l$与直线$m$平行,则过$m$的平面要么与$l$平行,要么经过$l$。
\end{tm}
\begin{proof2}
    给定过$m$的平面$\alpha$,$m\subset \alpha$与$l$平行,所以根据定理\ref{tm:1-0-20},
    $l$平行于平面$\alpha$或在$\alpha$中。
\end{proof2}

直线与平面的平行关系,也有传递性。
\begin{tm}
    若直线$l_1$与直线$l_2$平行,且与平面$\gamma$平行,则$l_2$与$\gamma$平行或在$\gamma$中。
\end{tm}
\begin{proof2}
    $l_1\parallel \gamma$。过$\gamma$中任一点作直线$n\parallel l_1$,则根据定理\ref{tm:1-0-40},
    $n$在$\gamma$中。$n\parallel l_1$,$l_1\parallel l_2$,所以根据定理\ref{tm:1-0-10},$n\parallel l_2$。
    根据定理\ref{tm:1-0-20},$l_2\parallel \gamma$或在$\gamma$中。
\end{proof2}

上面提到,两平面要么无公共点,要么相交,要么重合。我们把无公共点的平面称为平行平面,也用$\parallel$表记。
平面中,平行公理告诉我们,过直线外一点,恰有一条直线与之平行。空间中的平面,也有类似的结论:
\begin{tm}
    过平面外一点,恰有一平面与之平行。
\end{tm}\label{tm:1-0-50}
\begin{proof2}
    设平面$\alpha$外有点$P$。在$\alpha$上选一点$A$,过$A$作两相交直线$l,m$(交点为$A$)。
    根据平行公理,过$P$恰有直线$l', m'$分别与$l,m$平行。$l',m'$相交于点$P$,确定平面$\alpha'$。
    下面证明$\alpha,\alpha'$无公共点,即$\alpha'\parallel\alpha$。

    反设$\alpha,\alpha'$有公共点。由于$P\in\alpha'$在$\alpha$外,两者不重合。
    因此根据交面公理,$\alpha \cap \alpha'$是一条直线,记为$n$。$l,m,n$共面,$l,m$相交,
    因此$l,m$中至少有一条与$n$相交。设$l$与$n$相交,交点为$Q$,则$Q\in n\subset\alpha'$。又因为$l\parallel l'$,$Q\in l$,所以$Q\notin l'$,
    因而在$\alpha'$中,过$Q$可作$l'$的平行线。但这条线在$\alpha'$中,因此不是$l$。这与平行公理矛盾。
        
    因此,$\alpha,\alpha'$无公共点,$\alpha'\parallel\alpha$。
\end{proof2}

类似的结论还有:

\begin{tm}\label{tm:1-0-60}
    过平行于平面$\alpha$的一直线$l$,恰有一平面与$\alpha$平行。
\end{tm}
\begin{proof2}
在$l$上任取一点$P$,$P\notin \alpha$,因此根据定理\ref{tm:1-0-50},过$P$恰有一平面$\beta$与$\alpha$平行,只需证明$\beta$经过$l$。在$\alpha$中任取一点$Q$,则根据定理\ref{tm:1-0-40},过$Q$平行于$l$的直线$m$在$\alpha$中。$\beta$与$\alpha$平行,也就是说$\beta$与$\alpha$无公共点,所以$\beta$与$\alpha$的子集$m$也无公共点,即$m\parallel \beta$。过$P$作$m$的平行线,则根据定理\ref{tm:1-0-40},平行线在$\beta$中。而这条平行线就是$l$,所以$l$在$\beta$中。这说明过$l$恰有一平面$\beta$与$\alpha$平行。
\end{proof2}

从证明中,我们还可以提炼出判定平面平行(或重合)的准则:

\begin{tm}\label{tm:1-0-70}
    给定平面$\gamma_1, \gamma_2$。设$l,m$为$\gamma_1$中的相交直线。
    若$\gamma_2$中有直线$l',m'$分别与$l,m$平行或重合,则平面$\gamma_1, \gamma_2$平行或重合。
\end{tm}
\begin{proof2}
    两平面要么相互平行,要么重合,要么相交于一直线。反设$\gamma_1, \gamma_2$相交于直线$n$。

    如果$l=l'$,那么$l\subset \gamma_1\cap\gamma_2$,于是$n=l=l'$。设$m,l$交于点$P$,$m',l'$交于点$Q$。如果$P=Q$,那么$m=m'$,于是$\gamma_1$、$\gamma_2$都是$l,m$确定的平面,$\gamma_1=\gamma_2$。如果$P\neq Q$,那么$P\notin m'$。但$P\in l=n\subset\gamma_2$,因此过$P$作$m'$的平行线,平行线应该在$\gamma_2$中,因此根据平行公理,$m$在$\gamma_2$中。这说明$\gamma_1$、$\gamma_2$都是$l,m$确定的平面,$\gamma_1=\gamma_2$。于是总有两平面重合,矛盾。

    如果$l\parallel l'$,由于$l,m,n$共面,且$l,m$相交,
    因此$l,m$中至少有一条与$n$相交。设$l$与$n$相交,交点为$Q$,则$Q\in n\subset\gamma_2$。又因为$l\parallel l'$,$Q\in l$,所以$Q\notin l'$。
    在$\gamma_2$中,过$Q$可作$l'$的平行线。但这条线在$\gamma_2$中,因此不是$l$。这与平行公理矛盾。

    因此,平面$\gamma_1, \gamma_2$平行或重合。
\end{proof2}
平行平面之间,也有类似平行直线的传递性。
\begin{tm}\label{tm:1-0-80}

如果平面$\gamma_1,\gamma_2$都平行于平面$\beta$,那么$\gamma_1,\gamma_2$平行或重合。
\end{tm}
我们先证明一个小结论:
\begin{tm}\label{tm:1-0-90}
    设平面$\gamma_1\parallel \gamma_2$。平面$\beta$与$\gamma_1,\gamma_2$相交于直线$l_1,l_2$,
    则$l_1\parallel l_2$。
\end{tm}
\begin{proof2}
    一方面,$l_1,l_2$共面。另一方面,$\gamma_1\parallel \gamma_2$说明$l_1,l_2$无公共点。
    所以$l_1\parallel l_2$。
\end{proof2}
从这个结论还可以推出:如果平面$\gamma_1\parallel \gamma_2$,那么对$\gamma_1$中任意直线,过$\gamma_2$中任一点,作它的平行线,平行线都在$\gamma_2$中。

再来证明定理\ref{tm:1-0-80}。
\begin{proof2}
    已知平面$\gamma_1,\gamma_2$都平行于平面$\beta$。在$\beta$中找一点$P$,过$P$作相交直线$l,m$。在$\gamma_1$中找一点$Q$,过$Q$分别作$l,m$的平行线$l_1,m_1$,则$l_1,m_1$都在$\gamma_1$中。它们分别是平面$\gamma_1$与$l,Q$确定的平面$\alpha_1$、平面$\gamma_1$与$m,Q$确定的平面$\alpha_2$的交线。
    设$\alpha_1,\alpha_2$分别与$\gamma_2$交于$l_2,m_2$,由于$\gamma_2\parallel \beta$,
    所以根据定理\ref{tm:1-0-30},$l_2\parallel l$,$m_2\parallel m$。因此根据定理\ref{tm:1-0-10},
    $l_2$与$l_1$平行或重合,$m_2$与$m_1$平行或重合。根据定理\ref{tm:1-0-70},$\gamma_1$、$\gamma_2$平行或重合。
\end{proof2}

最后,我们还可以得到:
\begin{tm}\label{tm:1-0-100}
    若平面$\gamma_1$与直线$l$平行,且与平面$\gamma_2$平行,则$l$与$\gamma_2$平行或在$\gamma_2$中。
\end{tm}
\begin{proof2}
    $l\parallel \gamma_1$,所以过$l$恰有一平面$\beta$与$\gamma_1$平行。如果$\beta=\gamma_2$,
    则$l\subset\gamma_2$。如果$\beta\parallel\gamma_2$,那么$l\parallel \gamma_2$。
    如果$\beta$与$\gamma_2$相交于直线$m$,那么由于$m,l$共面且无公共点,$m\parallel l$。
    于是,根据定理\ref{tm:1-0-20},$l\parallel \gamma_2$或在$\gamma_2$中。
\end{proof2}

总结:

我们初步建立了关于空间形状的公理体系,引入了空间中平面的概念,并界定了点、直线和平面的关系:
\begin{enumerate}
    \item 直线和平面都是点的集合。
    \item 直线可能与平面平行、相交,或在平面中。
    \item 直线可能与直线异面、平行、相交、重合。
    \item 平面可能与平面平行、相交、重合。
    \item 直线与直线相交于一点,直线与平面相交于一点,平面与平面相交于一直线。
\end{enumerate}


\begin{sk}
    \mbox{} \\
    \indent 1. 定理\ref{tm:1-0-100}的证明中,我们讨论了$\beta$与$\gamma_2$相交于直线$m$的情形。实际上$\beta$是否会与$\gamma_2$相交?如何看待这个论证?
\end{sk}

\begin{xt}
    \mbox{} \\
    \indent 1. 如果直线$l$与直线$m$平行,那么过$l$的平面与过$m$的平面要么平行,要么重合,
    要么交于$l,m$之一,要么交于与$l,m$都平行的直线$n$。 
\end{xt}


\section{空间向量}

上一节中,我们使用公理体系讨论空间中的形状。可以看到,使用公理体系讨论虽然严谨,但过于抽象,步骤繁琐。仅处理点线面的平行关系,就需要大量的篇幅。为此,我们尝试用向量的概念来讨论空间中的形状。

平面中,我们用向量表示点,以及原点到此点的平移。现在我们把向量的概念扩展到空间中。我们把空间看作集合,记为$\mathbb{V}$,其中的元素称为向量或点。向量满足如下规则:

\begin{enumerate}
    \item 加法结合律:$\forall \,\, \mathbf{a}, \mathbf{b}, \mathbf{c} \in \mathbb{V}$,$\mathbf{a}+ (\mathbf{b} + \mathbf{c}) = (\mathbf{a} + \mathbf{b}) + \mathbf{c}$。
    \item 加法交换律:$\forall \,\, \mathbf{a}, \mathbf{b} \in \mathbb{V}$,$\mathbf{a} + \mathbf{b} = \mathbf{b} + \mathbf{a}$。
    \item 存在零向量:$\forall \,\, \mathbf{a} \in \mathbb{V}$,$\mathbf{a} + \mathbf{0} = \mathbf{a}$。
    \item 放缩和四则运算相容:$\forall \,\, \mathbf{a} \in \mathbb{V}$,$1\cdot \mathbf{a} = \mathbf{a}$。$\forall s, t \in \mathbb{R}$,$(s + t)\cdot\mathbf{a} = (s\cdot\mathbf{a}) + (t\cdot\mathbf{a})$,$(s \cdot t)\cdot \mathbf{a} = s \cdot (t\cdot \mathbf{a})$。
    \item 放缩和平移相容:$\forall \,\, \mathbf{a}, \mathbf{b} \in \mathbb{V}$,$\forall \,\, t \in \mathbb{R}$,$t\cdot(\mathbf{a} + \mathbf{b}) = t\cdot\mathbf{a} + t\cdot\mathbf{b}$。
\end{enumerate}

这个定义与平面向量相同。在此基础上,我们用同样的方式定义直线、线段和射线。
\begin{df}
    过原点的直线是非零向量放缩得到的集合。不过原点的直线是过原点的直线按一点平移得到的集合。
\end{df}
给定非零向量$A = \mathbf{a}$,$ \{t\mathbf{a} \, | \, t\in\mathbb{R}\}$是一条过原点$O$和$A$的直线$OA$,
称为$A$\textbf{引出}的直线,记为$\mathbb{R}\mathbf{a}$。
给定向量$B = \mathbf{b}$,$ \{t\mathbf{a}+\mathbf{b} \, | \, t\in\mathbb{R}\}$是一条过$B$的直线,
记为$\mathbb{R}\mathbf{a}+\mathbf{b}$,其中$\mathbb{R}\mathbf{a}$称为它的线性部分;
而$ \{t\mathbf{a}+(1 - t)\mathbf{b} \, | \, t\in\mathbb{R}\}$就是直线$AB$。

给定非零向量$\mathbf{a}$,如果向量$\mathbf{b}$可以通过$\mathbf{a}$放缩得到,
或者说$\mathbf{b}\in \mathbb{R}\mathbf{a}$,就称两者\textbf{共线}。

类比可以定义线段和射线:给定非零向量$A = \mathbf{a}$和向量$B =\mathbf{b}$,
$ \{(1 - t)\mathbf{a}+t\mathbf{b} \, | \, t\in [0, 1]\}$是线段$AB$,
$ \{(1 - t)\mathbf{a}+t\mathbf{b} \, | \, t \geqslant 0 \}$是射线$AB$。

空间向量与平面向量的唯一不同的地方在于,空间向量遵循的不再是平面的根本性质,
而是空间的根本性质。为了描述空间的根本性质,我们首先引进线性相关的概念:
\begin{df}
    给定$n$个向量$\mathbf{a}_1, \mathbf{a}_2, \cdots , \mathbf{a}_n$,
    对实数$t_1, t_2, \cdots , t_n$来说,
    $t_1\mathbf{a}_1 + t_2\mathbf{a}_2 + \cdots + t_1\mathbf{a}_n$称为这$n$个向量的\textbf{线性组合}。
    如果存在一组不全为零的实数$t_1, t_2, \cdots , t_n$,
    使得线性组合$t_1\mathbf{a}_1 + t_2\mathbf{a}_2 + \cdots + t_1\mathbf{a}_n$是零向量,
    就说这$n$个向量\textbf{线性相关}。如果不存在这样一组实数,就说这$n$个向量\textbf{线性无关}。
\end{df}

举例来说,单个非零向量总是线性无关的,因为非零实数乘以非零向量总得到非零向量。
又如:如果有不全为零的实数$t_1,t_2$使得向量$A, B$的线性组合:$t_1A + t_2B$等于零向量,就说$A, B$线性相关。

具体来说,设$A = 2\mathbf{e}_1 - \mathbf{e}_2$,$B = -6\mathbf{e}_1 + 3 \mathbf{e}_2$,那么
$$3A + B = \mathbf{0}.$$
也就是说,选取$t_1 = 3$,$t_2 = 1$,就使得线性组合:$t_1A + t_2B$等于零向量。因此,以上两个向量$A, B$线性相关。

设$A = 2\mathbf{e}_1 - \mathbf{e}_2$,$B = \mathbf{e}_1 + 3 \mathbf{e}_2$,那么对任何$t_1,t_2$,线性组合$t_1A + t_2B$可以写为:
\begin{align}
t_1A + t_2B &= t_1 \left( 2\mathbf{e}_1 - \mathbf{e}_2 \right) + t_2 \left( \mathbf{e}_1 + 3 \mathbf{e}_2 \right) \notag \\
&= (2t_1 + t_2) \mathbf{e}_1 + (-t_1 + 3t_2)\mathbf{e}_2.\notag
\end{align}
要使得$t_1A + t_2B$为零向量$(0,0)$,就要求它的横坐标和纵坐标同时为零。也就是说,$t_1,t_2$应该是一元一次方程组
$$
\left\{
\begin{array}{cl}
  2t_1 + t_2 &= 0 \\
  -t_1 +3t_2 &= 0 \\
\end{array}
\right.
$$
的解。解这个一元一次方程组,得到$t_1=t_2=0$。也就是说,不存在不全为零的实数$t_1, t_2$,
使得线性组合$t_1A + t_2B$为零向量。我们说$A, B$线性无关。

以上的例子也给出了判断一组向量是否线性相关的方法。
我们将“线性组合$t_1\mathbf{a}_1 + t_2\mathbf{a}_2 + \cdots + t_1\mathbf{a}_n$是零向量”
的条件转化为关于$t_1, t_2, \cdots , t_n$的一元一次方程组。如果方程组的解集中有不全为零的解,
这组向量就线性相关。如果方程组没有不全为零的解,就说这组向量线性无关。

直观来看,两个平面向量线性相关和共线是一回事。$A, B$线性相关,
就是说有有不全为零的实数$t_1,t_2$使得$t_1A + t_2B$等于零向量。
不妨设$t_1$不为零,那么$A = \frac{t_2}{t_1}B$,因此$A\in\mathbb{R}B$,即$A, B$共线。
反之亦然。平面的根本性质告诉我们,存在两个不共线的向量,也就是说,
不多于两个平面向量,可以线性无关。

如果向量多于两个,平面的根本性质告诉我们,只要其中两个向量$A, B$不共线,其余的向量都可以都可以表示成$sA + tB$的形式。因此,设有平面向量$\mathbf{a}_1, \mathbf{a}_2, \cdots , \mathbf{a}_n$。如果$\mathbf{a}_1, \mathbf{a}_2$不共线,那么它们线性相关,存在$t_1\mathbf{a}_1 + t_2\mathbf{a}_2 = \mathbf{0}$。于是
$$t_1\mathbf{a}_1 + t_2\mathbf{a}_2 + \sum_{i>2}0\cdot\mathbf{a}_i = \mathbf{0}.$$
如果$\mathbf{a}_1, \mathbf{a}_2$不共线,那么根据平面的根本性质,$\mathbf{a}_3$可以写成:
$$ \mathbf{a}_3 = s\mathbf{a}_1 + t\mathbf{a}_2.$$
于是
$$s\mathbf{a}_1 + t\mathbf{a}_2 - 1\cdot\mathbf{a}_3 + \sum_{i>3}0\cdot\mathbf{a}_i = \mathbf{0}.$$
也就是说,多于两个平面向量总线性相关。

这个结论反映了平面向量的本质:可以选出两个向量,所有向量都可以从它们开始,通过平移、放缩得到。
这两个向量叫作平面的基底。而空间中的点显然不一定在同一个平面里。我们把平面的根本性质替换为\textbf{空间的根本性质}:

\begin{enumerate}
    \item 给定一个非零向量,总能找到另一个向量,使得两者线性无关。
    \item 给定两个线性无关的向量,总能找到另一个向量,使得三者线性无关。
    \item 从线性无关的向量$A, B, C$出发,经过放缩、平移,可以得到空间中任何向量。具体来说,
    任何向量都可以表示成$sA + tB + uC$的形式,集合$\{sA + tB + uC \, | \, s, t, u \in\mathbb{R}\}$就是整个空间。
    这样的$A, B, C$称为空间的一组\textbf{基}或\textbf{基底}。
\end{enumerate}

对比平面和空间的根本性质,可以发现,主要的变化是“$2$变成$3$”。平面中保证存在两个线性无关的向量,
空间中保证存在三个线性无关的向量。按照类似的推理,我们可以得到结论:不多于三个空间向量,
可以线性无关;多于三个空间向量,总是线性相关。我们把这个数字称为\textbf{维数}。平面的维数是$2$,立体空间的维数是$3$。

与平面向量一样,给定基底后,任一空间向量$\mathbf{a}$可以唯一地写成基向量的线性组合:
$$ \mathbf{a} = a_x\mathbf{e}_x + a_y\mathbf{e}_y + a_z\mathbf{e}_z.$$
其中$a_x, a_y, a_z$是实数。$(a_x, a_y, a_z)$称为$\mathbf{a}$的\textbf{坐标},
$a_x, a_y, a_z$称为它的\textbf{坐标分量}。

于是我们可以定义立体空间中的平面:
\begin{df}
    过原点的平面是两个线性无关的向量通过平移、放缩得到的集合。不过原点的平面是过原点的平面按一点平移得到的集合。
\end{df}

给定线性无关的向量$A = \mathbf{a}, B = \mathbf{b}$,$ \{s\mathbf{a} + t\mathbf{b} \, | \, s, t\in\mathbb{R}\}$是一个过原点$O$和$A, B$的平面$OAB$。
给定向量$C = \mathbf{c}$,$ \{s\mathbf{a}+t\mathbf{b}+\mathbf{c}\, | \, s,t\in\mathbb{R}\}$是一条过$C$的直线;
而$ \{s\mathbf{a}+t\mathbf{b}+(1 - s - t)\mathbf{c} \, | \, t\in\mathbb{R}\}$就是过$A,B,C$的平面$ABC$。

来看几个具体的例子。选定空间的一组基底$\mathbf{e}_x,\mathbf{e}_y,\mathbf{e}_z$,
我们可以构建坐标系$Oxyz$,其中$x$轴、$y$轴、$z$轴分别是基向量引出的直线$\mathbb{R}\mathbf{e}_x$、
$\mathbb{R}\mathbf{e}_y$、$\mathbb{R}\mathbf{e}_z$。空间中任一点$A$可以写成$(a_x,a_y,a_z)$。
$x$轴、$y$轴构成平面:
$$ Oxy : \{s\mathbf{e}_x+t\mathbf{e}_y\,|\,s,t\in\mathbb{R}\}. $$
$x$轴、$z$轴构成平面:
$$ Oxz : \{s\mathbf{e}_x+t\mathbf{e}_z\,|\,s,t\in\mathbb{R}\}. $$
$y$轴、$z$轴构成平面:
$$ Oyz : \{s\mathbf{e}_y+t\mathbf{e}_z\,|\,s,t\in\mathbb{R}\}. $$
给定点$A(1,0,0)$、$B(0,1,0)$、$C(0,0,1)$,则经过它们的平面为:
$$ \{sA+tB+(1 - s - t)C \, | \, t\in\mathbb{R}\} = \{(s,\,t,\,1-s-t) \, | \, t\in\mathbb{R}\}. $$

可以验证,这样定义的点、直线、平面符合上一节中的各个公理(见附录B)。

\section{距离、长度和角度}
我们可以通过向量引进空间中距离和长度的概念。和平面中一样,
我们选定空间的一组基底$\mathbf{e}_x,\mathbf{e}_y,\mathbf{e}_z$,然后定义內积:
$$
\begin{array}{c}
    \forall  A = a_x\mathbf{e}_x+a_y\mathbf{e}_y+a_z\mathbf{e}_z,\,\,B = b_x\mathbf{e}_x+b_y\mathbf{e}_y+b_z\mathbf{e}_z, \notag \\
     A\cdot B = a_xb_x + a_yb_y + a_zb_z. \notag
\end{array}
$$ 
如果向量$A,B$的內积等于$0$,就说它们垂直。我们定义向量$A$的长度为
$$\|A\| = \sqrt{A\cdot A} = \sqrt{a_x^2 + a_y^2 + a_z^2},$$
长度为$1$的向量称为单位向量。任何非零向量除以自己的长度,都得到一个与自己共线的单位向量。
我们把这个操作称为\textbf{向量的归一}。

两点$A,B$之间的距离就是
$$\|A-B\| = \sqrt{(A - B)\cdot(A-B)} = \sqrt{(a_x - b_x)^2 + (a_y-b_y)^2 + (a_z-b_z)^2}. $$

平面向量的内积,小于等于长度之积。空间向量也有类似的性质:
\begin{align}
    &(a_x^2 + a_y^2 + a_z^2)(b_x^2 + b_y^2 + b_z^2) \notag \\
    =\,\,& (a_xb_x + a_yb_y + a_zb_z)^2 + (a_xb_y - a_yb_x)^2 + (a_xb_z - a_zb_x)^2 + (a_yb_z - a_zb_y)^2 \notag \\
    \geqslant\,\,& (a_xb_x + a_yb_y + a_zb_z)^2 \notag 
\end{align}
这个不等式也称为\textbf{内积不等式}。由此,类比平面向量,我们可以定义\textbf{空间向量的夹角}:
$$ \cos \angle AOB = \frac{a_xb_x + a_yb_y + a_zb_z}{\sqrt{(a_x^2 + a_y^2 + a_z^2)(b_x^2 + b_y^2 + b_z^2)}}$$
直线是非零向量放缩的结果,所以,我们可以定义空间中两条直线的夹角为引出它们的向量的夹角。

向量$A,B$垂直时,$\cos \angle AOB = 0$,即夹角为$90^\circ$。
向量夹角为$0^\circ$、$180^\circ$时,两向量共线,两直线同向或反向。
要注意的是,空间中,我们无法定义两向量夹角的方向。

\begin{sk}
    \mbox{} \\
    \indent 1. 内积不等式取等号的条件是什么?如何从直观上理解?\\
    \indent 2. 平面中,向量夹角的正弦与向量长度的乘积对应着向量的面积。
    立体空间中,是否可以作类似的定义?如何从直观上理解?
\end{sk}

\begin{xt}
    \mbox{} \\
    \indent 1. 已知向量$A(-1, 3, 1)$、$B(1, 2, 0)$,求它们的长度、内积和夹角。它们是否垂直?\\
    \indent 2. 已知向量$A(-2.4, 0, 1)$、$B(0.5, 1, 1.2)$,求它们的长度、内积和夹角。它们是否垂直?\\
    \indent 3. 已知直线$l_1: \mathbb{R}(1,2,-2.5) + (0,-1,-1)$、$l_2: \mathbb{R}(2,0,0.8) + (-2.5,1.1,1.7)$,
    求两直线的夹角。它们是否垂直?是否有公共点?\\
    \indent 4. 已知向量$A(2, 0, 1)$,求与$A$夹角为$60^\circ$的单位向量。
\end{xt}

\begin{appendix}

\chapter{空间形的表示法}

\chapter{空间向量与公理}

我们来验证,根据向量定义的点、直线、平面符合平面公理、平直公理、交面公理和补充的平行公理。

首先来看平面公理。任取不共线三点$A,B,C$,则$A-C,B-C$线性无关,
因此$ \{sA+tB+(1 - s - t)C \, | \,s, t\in\mathbb{R}\}$是$A,B,C$确定的平面。

如果点$P,Q$在平面$ \{sA+tB+(1 - s - t)C \, | \, t\in\mathbb{R}\}$中,那么存在$s_1, t_1, s_2, t_2$使得
\begin{align}
P &= s_1A + t_1B + (1 - s_1 - t_1)C \notag \\
Q &= s_2A + t_2B + (1 - s_2 - t_2)C \notag 
\end{align}
直线$PQ$是集合$\{uP+(1-u)Q \, | \, u\in \mathbb{R}\}$,其中任一点$U = uP+(1-u)Q$可以用$A,B,C$表示为
\begin{align}
U &= uP+(1-u)Q \notag \\
&= (us_1 + (1-u)s_2)A + (ut_1 + (1-u)t_2)B + (u(1 - s_1 - t_1) + (1-u)(1 - s_2 - t_2))C  \notag \\
&= (us_1 + (1-u)s_2)A + (ut_1 + (1-u)t_2)B + (1 - (us_1 + (1-u)s_2) - (ut_1 + (1-u)t_2))C \notag
\end{align}
因此$U$在平面$ABC$中。这说明向量表示的空间符合平直公理。

如果两平面$\gamma_1, \gamma_2$交于点$C$,记$\gamma_1: \{sA_1+tB_1+C\, | \, s,t\in\mathbb{R}\}$,$\gamma_2: \{sA_2+tB_2+C\, | \, s,t\in\mathbb{R}\}$。
其中$(A_1,B_1)$线性无关,$(A_2,B_2)$线性无关。两平面公共点可以表示为让以下等式成立的$s_1,t_1,s_2,t_2$:
$$ s_1A_1+t_1B_1+C = s_2A_2+t_2B_2+C. $$
以上等式可以转为:
$$ s_1A_1+t_1B_1 - s_2A_2 - t_2B_2 = \mathbf{0}. $$
根据空间的根本性质,四个向量总是线性相关。所以存在不全为零的四个实数$s_1,t_1,s_2,t_2$使上式成立。
由于$(A_1,B_1)$线性无关,$(A_2,B_2)$线性无关,所以$s_1A_1+t_1B_1$和$s_2A_2 + t_2B_2$为零向量时,
$s_1,t_1,s_2,t_2$必然全为零。然而$s_1,t_1,s_2,t_2$不全为零,所以$s_1A_1+t_1B_1$和$s_2A_2 + t_2B_2$不为零向量。
于是这时$s_1A_1+t_1B_1+C = s_2A_2+t_2B_2+C\neq C$是两平面另一个公共点。这说明向量表示的空间符合交面公理。

平面中,两条直线平行,当且仅当它们的线性部分是同一条过原点的直线。我们用这个方法来判定空间中直线的平行关系。设有两直线$l_1\parallel l_2$,那么它们是同一条过原点的直线平移而成:
\begin{align}
l_1: & \,\,\, \{t\mathbf{a} + \mathbf{b}_1 \, | \, t\in\mathbb{R}\} \notag \\
l_2: & \,\,\, \{t\mathbf{a} + \mathbf{b}_2 \, | \, t\in\mathbb{R}\} \notag
\end{align}
所以$l_1,l_2$都在平面:$\gamma : \{t\mathbf{a} + s(\mathbf{b}_1 - \mathbf{b}_2) + \mathbf{b}_2 \, | \, s,t\in\mathbb{R}\}$中。对任意$t\in\mathbb{R}$,$l_1$中的点$t\mathbf{a} + \mathbf{b}_1 = t\mathbf{a} + 1\cdot(\mathbf{b}_1 - \mathbf{b}_2) + \mathbf{b}_2 \in \gamma$;$l_2$中的点$t\mathbf{a} + \mathbf{b}_2 = t\mathbf{a} + 0\cdot(\mathbf{b}_1 - \mathbf{b}_2) + \mathbf{b}_2 \in \gamma$。
这说明,两条平行直线总在同一平面内,即向量表示的空间符合补充后的平行公理。

\end{appendix}





\end{document}